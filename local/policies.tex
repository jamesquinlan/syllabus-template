% Fri Jun 09 10:16:32 EDT 2023
%+++++++++++++++++++++++++++++++++++++++++++++++++++++++++++++++
% SUMMARY   :  Policies, Support Services, and Other Information
%           :  
%+++++++++++++++++++++++++++++++++++++++++++++++++++++++++++++++

% Attendance
\subsection*{Attendance}
 
Requirements for students’ attendance and participation will be defined by each instructor based on the following policy:
\begin{outline}
	\1	Tuesday of the first week is considered the first day of class for the online term, with subsequent Mondays serving as the first day of instruction each week. Assignments each week are due on Thursdays and/or Sundays. Please see the course schedule for specific due days.
	\1	Regular online attendance/participation and engagement are expected for student success in fully online courses. Online participation is evident through posting to a discussion board, completing real-time activities or quizzes, or other course-related activities.
	\1	Weekly deadlines for deliverables are Thursdays and Sundays by 11:59 pm.
\end{outline}
\vspace{2ex}\hrule\vspace{2ex}




% Brightspace
\subsection*{Brightspace Support Services}
The ITS Service Desk in the ABC Library is prepared to help students should they encounter problems with Brightspace. Please read through the following information:
\begin{outline}
	\1	For login problems, call the Service Desk at 874-4357.
	\1	The Service Desk Website, \url{https://web.ABC.edu/itservicedesk} opens in a new window and posts the semester operating schedule and a link on the right index to the self-help technical wiki. That site contains Brightspace help and instructions for both students and faculty.
\end{outline}


Recommended browsers (those with the most QA testing effort against them) are Google Chrome, Safari, and Mozilla Firefox. The mobile versions of these browsers also work well with most operations in Brightspace. Internet Explorer is not recommended.

\vspace{2ex}\hrule\vspace{2ex}




% Netiquette
\subsection*{Netiquette for Online Courses}
\begin{outline}
	\1	Be polite and respectful of one another.
	\1	Avoid personal attacks. Keep dialogue friendly and supportive, even when you disagree or wish to present a controversial idea or response.
	\1	Be careful with the use of humor and sarcasm. Emotion is difficult to sense through text.
	\1	Be helpful and share your expertise. Foster community communication and collaboration.
	\1	Contribute constructively and completely to each discussion. Avoid short repetitive ``I agree" responses, and don’t make everyone else do the work.
	\1	Consider carefully what you write. Re-read all e-mails and discussions before sending or posting.
	\1	e-mail is considered a permanent record that may be forwarded to others.
	\1	Be brief and concise. Don’t use up other people’s time or bandwidth.
	\1	Use descriptive subject headings for each e-mail message.
	\1	Respect privacy. Don’t forward a personal message without permission.
	\1	Cite references. Include web addresses, authors, articles' names, publication date, etc.
	\1	Keep responses professional and educational. Do not advertise or send chain letters.
	\1	Do not send large attachments unless you have been requested to do so or have permission from all parties.
	\1	2-word postings (e.g., I agree, Oh yeah, No way, Me too) do not ``count'' as postings.
\end{outline}

\vspace{2ex}\hrule\vspace{2ex}



% Writing Support
\subsection*{ABC Academic Writing Standards}
Specific writing standards differ from discipline to discipline, and learning to write persuasively in any genre is a complex process, both individual and social, that takes place over time with continued practice and guidance.  Nonetheless, ABC has identified some common assumptions and practices that apply to most academic writing done at the university level.  These generally understood elements are articulated here to help students see how they can best express their ideas effectively, regardless of their discipline or writing assignment. \\
 
Venues for writing include the widespread use of e-mail, electronic chat spaces, and interactive blackboards.  ABC is committed to guaranteeing that students can expect all electronic communication to meet Federal and State regulations concerning harassment or other ``hate'' speech. Individual integrity and social decency require common courtesies and a mutual understanding that writing- in all its educational configurations- attempts to share information, knowledge, opinions, and insights in fruitful ways. \\

Academic writing (as commonly understood in the university) always aims at correct Standard English grammar, punctuation, and spelling.\\

The following details are meant to give students accurate, useful, and practical assistance for writing across the curriculum of ABC.  \\

Students can assume that successful collegiate writing will generally:  
\begin{outline}
	\1	Delineate the relationships among writer, purpose, and audience using a clear focus (thesis statements, hypotheses, or instructor-posed questions are examples of such focusing methods, but are not the only ones) and a topic managed and developed appropriately for the specific task. 
	\1	Display a familiarity with and understanding of the particular discourse styles of the discipline and/or particular assignment.
	\1	Demonstrate the analytical skills of the writer rather than just repeating what others have said by summarizing or paraphrasing
	\1	Substantiate abstractions, judgments, and assertions with evidence specifically applicable to the occasion, whether illustrations, quotations, or relevant data.
	\1	Draw upon contextualized research whenever necessary, properly acknowledging the explicit work or intellectual property of others.
	\1	Require more than one carefully proofread and documented draft, typed or computer printed unless otherwise specified.
\end{outline}

\vspace{2ex}\hrule\vspace{2ex}



% Conduct
\subsection*{Professional Conduct}
Cheating and plagiarism are serious academic offenses that Colleges and Universities deal with firmly. Academic integrity presumes that students are honest in all academic work. Cheating is the failure to give credit for work not done independently (i.e., submitting a paper written by someone other than yourself), unauthorized communication dABCng an examination, or claiming credit for work not done (i.e., falsifying information). Plagiarism is the failure to give credit for another person’s written or oral statement, thereby falsely presuming that such work is originally and solely your own. \\ 

If you have any doubt about what constitutes plagiarism, visit the following website:\\

 \url{https://honorcouncil.georgetown.edu/whatisplagiarism}\\
 
 the ABC Student Handbook, and University Manual sections on plagiarism and cheating at\\
 
\url{http://web.ABC.edu/studentconduct/student-handbook}.  \\


Students are expected to be honest in all academic work. A student’s name on any written work, quiz, or exam shall be regarded as assurance that the work results from the student’s independent thought and study. Work should be stated in the student’s words and correctly attributed to its source. Students should know how to quote, paraphrase, summarize, cite, and reference the work of others with integrity. The following are examples of academic dishonesty.
\begin{outline}
	\1	Using material, directly or paraphrasing, from published sources (print or electronic) without appropriate citation;
	\1	Claiming disproportionate credit for work not done independently;
	\1	Unauthorized possession or access to exams;
	\1	Unauthorized communication during exams;
	\1	Unauthorized use of another’s work or preparing work for another student;
	\1	Taking an exam for another student;
	\1	Altering or attempting to alter grades;
	\1	The use of notes or electronic devices to gain an unauthorized advantage during exams;
	\1	Fabricating or falsifying facts, data, or references;
	\1	Facilitating or aiding another’s academic dishonesty;
	\1	Submitting the same paper for more than one course without prior approval from the Instructor.
\end{outline}



Please note the following section from the University Manual:
\textbf{8.27.17.} Instructors shall have the explicit duty to act in known cheating or plagiarism cases. The instructor shall have the right to fail a student on the assignment on which the instructor has determined that a student has cheated or plagiarized. The circumstances of this failure shall be reported to the student’s academic dean, the instructor’s dean, and the Office of Student Life. The student may appeal the matter to the instructor’s dean, and the decision by the dean shall be expeditious and final.
The instructor will initiate such action if it is determined that any written assignment is copied or falsified, or inappropriately referenced.\\

Any good writer’s handbook and reputable online resources will offer help on plagiarism and instruct you on how to acknowledge source material. If you need more help understanding when to cite something or how to indicate your references, PLEASE ASK.\\

Please note:  Students are responsible for being familiar with and adhering to the published ``Community Standards of Behavior: University Policies and Regulations'' which can be accessed in the University Student Handbook.\\

 \vspace{2ex}\hrule\vspace{2ex}

% Support
\subsection*{Student Support Services}
The following student support services are provided by the university and available to all ABC students: 
\begin{outline}
	\1	Student support services such as counseling center: https://web.ABC.edu/counseling 
	\1	Food assistance: https://web.ABC.edu/rhody-outpost 
	\1	Bias resource team: https://web.ABC.edu/brt 
\end{outline}

\vspace{2ex}\hrule\vspace{2ex}





% Support
\subsection*{Academic Support Services}

\textbf{Office of Disability Services}\\
\underline{Americans With Disabilities Act Statement}\\

Any personal learning accommodations that may be needed by a student covered by the ``Americans with Disabilities Act'' must be made known to the university as soon as possible.  This is the student's responsibility. The Office of Affirmative Action, Equal Opportunity and Diversity (AAEOD) can provide information about services, academic modifications, and documentation requirements. \url{https://web.ABC.edu/affirmativeaction/}\\


Any student with a documented disability is welcome to contact me early in the semester so that we may work out reasonable accommodations to support your success in this course. Students should also contact Disability Services for Students, Office of Student Life, 123 Student Union, 123-456-7890.\\


Students are expected to notify faculty at the onset of the semester if any special considerations are required in the classroom. If any special considerations are required for examinations, the student is expected to notify the faculty a week before the examination with the appropriate paperwork.

 \vspace{2ex}\hrule\vspace{2ex}



% Library
\subsection*{ABC Online Library Resources}
\url{https://web.ABC.edu/library/}


